\documentclass[12pt,a4paper]{article}
\usepackage[utf8]{inputenc}
\usepackage[T2A]{fontenc}
\usepackage[russian]{babel}
\usepackage{hyperref}
\usepackage{listings}
\usepackage{xcolor}
\usepackage{graphicx}

\definecolor{codegreen}{rgb}{0,0.6,0}
\definecolor{codegray}{rgb}{0.5,0.5,0.5}
\definecolor{codepurple}{rgb}{0.58,0,0.82}
\definecolor{backcolour}{rgb}{0.95,0.95,0.95}

% Добавляем определение языка C#
\lstdefinelanguage{csharp}
{
  keywords={abstract, as, base, bool, break, byte, case, catch, char, checked, class, const, continue, decimal, default, delegate, do, double, else, enum, event, explicit, extern, false, finally, fixed, float, for, foreach, goto, if, implicit, in, int, interface, internal, is, lock, long, namespace, new, null, object, operator, out, override, params, private, protected, public, readonly, ref, return, sbyte, sealed, short, sizeof, stackalloc, static, string, struct, switch, this, throw, true, try, typeof, uint, ulong, unchecked, unsafe, ushort, using, virtual, void, volatile, while},
  sensitive=true,
  morecomment=[l]{//},
  morecomment=[s]{/*}{*/},
  morestring=[b]",
  morestring=[b]',
  morestring=[b]@",
}

\lstdefinestyle{mystyle}{
    backgroundcolor=\color{backcolour},
    commentstyle=\color{codegreen},
    keywordstyle=\color{blue},
    stringstyle=\color{codepurple},
    basicstyle=\ttfamily\footnotesize,
    breakatwhitespace=false,
    breaklines=true,
    captionpos=b,
    keepspaces=true,
    numberstyle=\tiny\color{codegray},
    numbersep=5pt,
    showspaces=false,
    showstringspaces=false,
    showtabs=false,
    tabsize=2
}

\lstset{style=mystyle}

\title{Отчет о разработке приложения для учета финансов "ВШЭ-Банк"}
\author{Эдуард Альтшуль}
\date{\today}

\begin{document}

\maketitle

\tableofcontents
\newpage

\section{Общая идея решения}

Разработанное приложение "ВШЭ-Банк" представляет собой программный комплекс для учета личных финансов. Основная цель проекта - создать надежную архитектуру с использованием современных принципов и паттернов проектирования. 

\subsection{Реализованный функционал}
\begin{itemize}
    \item Управление банковскими счетами (создание, редактирование, удаление)
    \item Работа с категориями доходов и расходов (создание, редактирование, удаление)
    \item Учет финансовых операций (доходы, расходы)
    \item Аналитические инструменты (статистика по категориям, периодам)
    \item Экспорт данных в различные форматы (CSV, JSON)
    \item Импорт данных (CSV, JSON, YAML)
    \item Мониторинг производительности операций
    \item Высокий процент покрытия кода тестами (более 65\%)
\end{itemize}

\section{Архитектура приложения}

Приложение спроектировано с использованием многослойной архитектуры, где четко разделены ответственности между слоями:

\begin{itemize}
    \item \textbf{Доменный слой} - содержит основные бизнес-сущности и их поведение
    \item \textbf{Сервисный слой} - реализует бизнес-логику приложения
    \item \textbf{Уровень представления} - консольный интерфейс пользователя
\end{itemize}

\subsection{Структура проекта}

\begin{lstlisting}
/FinanceApp
|-- /Domain
|   |-- BankAccount.cs
|   |-- Category.cs
|   |-- Operation.cs
|
|-- /Services
|   |-- /Facade
|   |   |-- BankAccountFacade.cs
|   |   |-- CategoryFacade.cs
|   |   |-- OperationFacade.cs
|   |   |-- AnalyticsFacade.cs
|   |
|   |-- /Command
|   |   |-- ICommand.cs
|   |   |-- CreateOperationCommand.cs
|   |   |-- UpdateBalanceCommand.cs
|   |   |-- DeleteOperationCommand.cs
|   |   |-- TimeMeasureDecorator.cs
|   |
|   |-- /Export
|   |   |-- IVisitor.cs
|   |   |-- CsvExportVisitor.cs
|   |   |-- JsonExportVisitor.cs
|   |
|   |-- /Import
|   |   |-- ImportTemplate.cs
|   |   |-- CsvImport.cs
|   |   |-- JsonImport.cs
|   |   |-- YamlImport.cs
|   |
|   |-- /Proxy
|   |   |-- BankAccountProxy.cs
|   |
|   |-- /Implementations
|       |-- FinancialObjectFactory.cs
|
|-- /Util
|   |-- NonClosingStringWriter.cs
|
|-- Program.cs
|-- usings.cs
|-- FinanceApp.csproj
|-- README.md
|
/FinanceApp.Tests
|-- BasicTests.cs
|-- BankAccountTests.cs
|-- CategoryTests.cs
|-- OperationTests.cs
|-- AnalyticsTests.cs
|-- PerformanceTests.cs
|-- FileOperationsTests.cs
|-- ImportExportInterfaceTests.cs
|-- MoreTests.cs
|-- YamlImportTests.cs
|-- DomainModelTests.cs
|-- ComplexIntegrationTests.cs
|-- FactoryTests.cs
|-- OperationDetailedTests.cs
|-- OperationFacadeTests.cs
|-- AnalyticsAdvancedTests.cs
|-- EdgeCaseTests.cs
\end{lstlisting}

\section{Принципы SOLID и GRASP в проекте}

\subsection{Принципы SOLID}

\subsubsection{S - Single Responsibility Principle (Принцип единственной ответственности)}

Этот принцип реализован во множестве классов проекта:

\begin{itemize}
    \item \textbf{Доменные классы} (BankAccount, Category, Operation) - отвечают только за свои данные и поведение без внешней логики.
    \item \textbf{Фасады} (BankAccountFacade, CategoryFacade, OperationFacade, AnalyticsFacade) - каждый из них отвечает только за операции с соответствующим типом объектов.
    \item \textbf{Экспортеры/Импортеры} (CsvExportVisitor, JsonExportVisitor, CsvImport, JsonImport) - каждый класс отвечает только за свой формат данных.
\end{itemize}

Пример из кода BankAccountFacade:

\begin{lstlisting}[language=csharp]
// BankAccountFacade is responsible only for bank account operations
public class BankAccountFacade
{
    private readonly Dictionary<int, BankAccount> _accounts = new Dictionary<int, BankAccount>();
    private readonly FinancialObjectFactory _factory;

    public BankAccountFacade(FinancialObjectFactory factory)
    {
        _factory = factory ?? throw new ArgumentNullException(nameof(factory));
    }

    public BankAccount CreateAccount(string name, decimal initialBalance, AccountType type = AccountType.Checking)
    {
        var account = _factory.CreateBankAccount(name, initialBalance, type);
        _accounts[account.Id] = account;
        return account;
    }
    
    // Other methods for account operations
}
\end{lstlisting}

\subsubsection{O - Open/Closed Principle (Принцип открытости/закрытости)}

Этот принцип реализован в нескольких компонентах:

\begin{itemize}
    \item \textbf{IVisitor и его реализации} - система экспорта позволяет добавлять новые форматы без изменения существующего кода.
    \item \textbf{ImportTemplate} - абстрактный шаблонный класс, закрытый для изменений, но открытый для расширения через наследование.
    \item \textbf{ICommand и его реализации} - команды для различных операций можно добавлять без изменения кода, использующего интерфейс.
\end{itemize}

Пример с реализацией паттерна Посетитель (Visitor):

\begin{lstlisting}[language=csharp]
// Visitor interface is open for new implementations
public interface IVisitor
{
    void Visit(BankAccount account);
    void Visit(Category category);
    void Visit(Operation operation);
}

// CsvExportVisitor - concrete implementation
public class CsvExportVisitor : IVisitor
{
    private readonly StringBuilder _sb = new StringBuilder();

    public void Visit(BankAccount account)
    {
        _sb.AppendLine($"BankAccount;{account.Id};{account.Name};{account.Balance}");
    }
    
    // Implementations for other types...
    
    public string GetCsvResult() => _sb.ToString();
}
\end{lstlisting}

\subsubsection{L - Liskov Substitution Principle (Принцип подстановки Лисков)}

Этот принцип реализован в классах, которые наследуют общие абстракции:

\begin{itemize}
    \item \textbf{ImportTemplate} и его наследники (CsvImport, JsonImport, YamlImport) - корректно реализуют абстрактные методы.
    \item \textbf{ICommand} и его реализации - любая команда может использоваться через интерфейс.
    \item \textbf{TimeMeasureDecorator} - корректно расширяет функциональность ICommand.
\end{itemize}

Пример с шаблонным методом:

\begin{lstlisting}[language=csharp]
// Base abstract class with template method
public abstract class ImportTemplate
{
    public void ImportFile(string path)
    {
        var fileContent = File.ReadAllText(path);
        var parsedData = ParseData(fileContent);
        ProcessData(parsedData);
    }

    protected abstract object ParseData(string fileContent);
    protected abstract void ProcessData(object data);
}

// Descendant correctly implements abstract methods
public class CsvImport : ImportTemplate
{
    protected override object ParseData(string fileContent)
    {
        return fileContent.Split('\n');
    }

    protected override void ProcessData(object data)
    {
        var lines = data as string[];
        foreach(var line in lines ?? Array.Empty<string>())
        {
            Console.WriteLine($"Importing CSV line: {line}");
        }
    }
}
\end{lstlisting}

\subsubsection{I - Interface Segregation Principle (Принцип разделения интерфейсов)}

Этот принцип применен к интерфейсам проекта:

\begin{itemize}
    \item \textbf{IVisitor} - содержит только необходимые методы для посещения различных типов объектов.
    \item \textbf{ICommand} - имеет минимально необходимый набор методов (только Execute).
    \item \textbf{IBankAccountProxy} - содержит только методы, необходимые для кэширования банковских счетов.
\end{itemize}

Пример интерфейса ICommand:

\begin{lstlisting}[language=csharp]
// Minimal command interface
public interface ICommand
{
    void Execute();
}
\end{lstlisting}

\subsubsection{D - Dependency Inversion Principle (Принцип инверсии зависимостей)}

Этот принцип реализован через:

\begin{itemize}
    \item \textbf{Внедрение зависимостей} - фасады и другие сервисы принимают необходимые зависимости через конструктор.
    \item \textbf{Зависимость от абстракций} - классы зависят от интерфейсов и абстрактных классов, а не конкретных реализаций.
    \item \textbf{DI-контейнер} - в Program.cs используется ServiceCollection для управления зависимостями.
\end{itemize}

Пример внедрения зависимостей:

\begin{lstlisting}[language=csharp]
// Class depends on abstraction (interface), not concrete implementation
public class OperationFacade
{
    private readonly Dictionary<int, Operation> _operations = new Dictionary<int, Operation>();
    private readonly FinancialObjectFactory _factory;
    private readonly BankAccountFacade _accountFacade;

    public OperationFacade(FinancialObjectFactory factory, BankAccountFacade accountFacade)
    {
        _factory = factory ?? throw new ArgumentNullException(nameof(factory));
        _accountFacade = accountFacade ?? throw new ArgumentNullException(nameof(accountFacade));
    }
    
    // Class methods...
}
\end{lstlisting}

\subsection{Принципы GRASP}

\subsubsection{High Cohesion (Высокая связность)}

\begin{itemize}
    \item \textbf{Доменные классы} - каждый класс имеет высокую связность, содержа только свои данные и методы.
    \item \textbf{Фасады} - группируют родственные операции (например, операции со счетами в BankAccountFacade).
    \item \textbf{AnalyticsFacade} - объединяет всю аналитическую функциональность в одном месте.
\end{itemize}

\subsubsection{Low Coupling (Низкая связанность)}

\begin{itemize}
    \item \textbf{Использование интерфейсов} - компоненты взаимодействуют через интерфейсы (ICommand, IVisitor).
    \item \textbf{Фасады} - служат точками входа и скрывают внутреннюю реализацию, снижая связанность.
    \item \textbf{Внедрение зависимостей} - зависимости явно передаются через конструктор.
\end{itemize}

\subsubsection{Creator (Создатель)}

\begin{itemize}
    \item \textbf{FinancialObjectFactory} - централизованно создает доменные объекты.
    \item \textbf{Фасады} - управляют созданием объектов через фабрику, у которой есть вся необходимая информация.
\end{itemize}

\subsubsection{Controller (Контроллер)}

\begin{itemize}
    \item \textbf{Program} - координирует действия на высоком уровне, делегируя их фасадам.
    \item \textbf{Фасады} - служат контроллерами для операций с соответствующими типами объектов.
\end{itemize}

\subsubsection{Information Expert (Информационный эксперт)}

\begin{itemize}
    \item \textbf{Доменные классы} - имеют всю информацию для вычисления своих свойств.
    \item \textbf{AnalyticsFacade} - собирает информацию из разных источников для расчета аналитики.
\end{itemize}

\section{Паттерны GoF в проекте}

\subsection{Порождающие паттерны}

\subsubsection{Factory (Фабрика)}

\textbf{Реализация:} Класс \textbf{FinancialObjectFactory} в файле \texttt{FinanceApp/Services/Implementations/FinancialObjectFactory.cs}

\textbf{Обоснование важности:} Фабрика централизует создание объектов, что позволяет:
\begin{itemize}
    \item Инкапсулировать логику создания объектов
    \item Обеспечить валидацию входных параметров
    \item Генерировать уникальные идентификаторы для объектов
    \item Упростить добавление новых типов объектов в будущем
\end{itemize}
    
\begin{lstlisting}[language=csharp]
public class FinancialObjectFactory
{
    private int _nextAccountId = 1;
    private int _nextCategoryId = 1;
    private int _nextOperationId = 1;

    public BankAccount CreateBankAccount(string name, decimal initialBalance, AccountType type = AccountType.Checking)
    {
        if (initialBalance < 0)
            throw new ArgumentException("Initial balance cannot be negative", nameof(initialBalance));
        
        return new BankAccount(_nextAccountId++, name, initialBalance, type);
    }

    // Methods for creating other objects...
}
\end{lstlisting}

\subsection{Структурные паттерны}

\subsubsection{Facade (Фасад)}

\textbf{Реализация:} Классы \textbf{BankAccountFacade}, \textbf{CategoryFacade}, \textbf{OperationFacade}, \textbf{AnalyticsFacade} в директории \texttt{FinanceApp/Services/Facade/}

\textbf{Обоснование важности:} Фасады предоставляют высокоуровневый интерфейс к подсистемам:
\begin{itemize}
    \item Скрывают сложность внутренней реализации
    \item Предоставляют единую точку входа для операций над определенным типом объектов
    \item Повышают читаемость и поддерживаемость кода
    \item Упрощают взаимодействие с доменными объектами
\end{itemize}

\begin{lstlisting}[language=csharp]
public class AnalyticsFacade
{
    private readonly OperationFacade _operationFacade;
    private readonly CategoryFacade _categoryFacade;

    public AnalyticsFacade(OperationFacade operationFacade, CategoryFacade categoryFacade)
    {
        _operationFacade = operationFacade ?? throw new ArgumentNullException(nameof(operationFacade));
        _categoryFacade = categoryFacade ?? throw new ArgumentNullException(nameof(categoryFacade));
    }

    public decimal CalculateIncomeExpenseDifference(DateTime start, DateTime end)
    {
        var totalIncome = _operationFacade.GetIncomeTotal(start, end);
        var totalExpense = _operationFacade.GetExpenseTotal(start, end);
        return totalIncome - totalExpense;
    }

    // Other analytics methods...
}
\end{lstlisting}

\subsubsection{Proxy (Прокси)}

\textbf{Реализация:} Класс \textbf{BankAccountProxy} в файле \texttt{FinanceApp/Services/Proxy/BankAccountProxy.cs}

\textbf{Обоснование важности:} Паттерн Прокси обеспечивает:
\begin{itemize}
    \item Кэширование объектов в памяти для ускорения доступа
    \item Отложенную загрузку данных
    \item Контроль доступа к объектам
    \item Повышение производительности приложения
\end{itemize}

\begin{lstlisting}[language=csharp]
public interface IBankAccountProxy
{
    BankAccount? GetById(int id);
    BankAccount Save(BankAccount account);
    bool Remove(int id);
}

public class BankAccountProxy : IBankAccountProxy
{
    private readonly Dictionary<int, BankAccount> _cache = new Dictionary<int, BankAccount>();

    public BankAccount? GetById(int id)
    {
        _cache.TryGetValue(id, out var acct);
        return acct;
    }

    public BankAccount Save(BankAccount account)
    {
        _cache[account.Id] = account;
        return account;
    }
    
    public bool Remove(int id)
    {
        return _cache.Remove(id);
    }
}
\end{lstlisting}

\subsection{Поведенческие паттерны}

\subsubsection{Command (Команда)}

\textbf{Реализация:} Интерфейс \textbf{ICommand} и его реализации в директории \texttt{FinanceApp/Services/Command/}

\textbf{Обоснование важности:} Паттерн Команда позволяет:
\begin{itemize}
    \item Инкапсулировать действия в объекты
    \item Параметризовать клиентов с разными запросами
    \item Создавать последовательности команд
    \item Реализовать транзакционность (возможность отмены операций)
    \item Разделить ответственность между объектами
\end{itemize}

\begin{lstlisting}[language=csharp]
public interface ICommand
{
    void Execute();
}

public class CreateOperationCommand : ICommand
{
    private readonly OperationFacade _facade;
    private readonly OperationType _type;
    private readonly int _accountId;
    private readonly decimal _amount;
    private readonly int _categoryId;
    private readonly string _description;

    // Constructor with parameters...

    public void Execute()
    {
        _facade.CreateOperation(_type, _accountId, _amount, DateTime.Now, _categoryId, _description);
    }
}
\end{lstlisting}

\subsubsection{Decorator (Декоратор)}

\textbf{Реализация:} Класс \textbf{TimeMeasureDecorator} в файле \texttt{FinanceApp/Services/Command/TimeMeasureDecorator.cs}

\textbf{Обоснование важности:} Паттерн Декоратор обеспечивает:
\begin{itemize}
    \item Динамическое добавление функциональности объектам без изменения их структуры
    \item Альтернативу наследованию для расширения функциональности
    \item Возможность комбинировать множество дополнительных поведений
    \item Соблюдение принципа открытости/закрытости (OCP)
\end{itemize}

\begin{lstlisting}[language=csharp]
public class TimeMeasureDecorator : ICommand
{
    private readonly ICommand _command;
    private readonly string _commandName;

    public TimeMeasureDecorator(ICommand command, string commandName = null)
    {
        _command = command;
        _commandName = commandName ?? command.GetType().Name;
    }

    public void Execute()
    {
        var stopwatch = Stopwatch.StartNew();
        
        try
        {
            _command.Execute();
        }
        finally
        {
            stopwatch.Stop();
            Console.WriteLine($"Command execution time {_commandName}: {stopwatch.ElapsedMilliseconds} ms");
        }
    }
}
\end{lstlisting}

\subsubsection{Template Method (Шаблонный метод)}

\textbf{Реализация:} Абстрактный класс \textbf{ImportTemplate} в файле \texttt{FinanceApp/Services/Import/ImportTemplate.cs} и его наследники

\textbf{Обоснование важности:} Паттерн Шаблонный метод позволяет:
\begin{itemize}
    \item Определить скелет алгоритма, делегируя конкретные шаги подклассам
    \item Избежать дублирования кода
    \item Обеспечить расширяемость алгоритма
    \item Сохранить контроль над последовательностью выполнения шагов алгоритма
\end{itemize}

\begin{lstlisting}[language=csharp]
// Base abstract class with template method
public abstract class ImportTemplate
{
    public void ImportFile(string path)
    {
        var fileContent = File.ReadAllText(path);
        var parsedData = ParseData(fileContent);
        ProcessData(parsedData);
    }

    protected abstract object ParseData(string fileContent);
    protected abstract void ProcessData(object data);
}

// Descendant correctly implements abstract methods
public class CsvImport : ImportTemplate
{
    protected override object ParseData(string fileContent)
    {
        return fileContent.Split('\n');
    }

    protected override void ProcessData(object data)
    {
        var lines = data as string[];
        foreach(var line in lines ?? Array.Empty<string>())
        {
            Console.WriteLine($"Importing CSV line: {line}");
        }
    }
}
\end{lstlisting}

\subsubsection{Visitor (Посетитель)}

\textbf{Реализация:} Интерфейс \textbf{IVisitor} и его реализации в директории \texttt{FinanceApp/Services/Export/}

\textbf{Обоснование важности:} Паттерн Посетитель обеспечивает:
\begin{itemize}
    \item Отделение алгоритмов от структуры объектов
    \item Добавление новых операций к классам без изменения их кода
    \item Выполнение разных операций над разными типами объектов
    \item Сбор связанных операций в одном классе
\end{itemize}

\begin{lstlisting}[language=csharp]
public interface IVisitor
{
    void Visit(BankAccount account);
    void Visit(Category category);
    void Visit(Operation operation);
}

public class JsonExportVisitor : IVisitor
{
    private readonly List<object> _entities = new List<object>();

    public void Visit(BankAccount account)
    {
        _entities.Add(new 
        {
            Type = "BankAccount",
            account.Id,
            account.Name,
            account.Balance
        });
    }

    // Other Visit methods...

    public string GetJsonResult()
    {
        return JsonSerializer.Serialize(_entities, new JsonSerializerOptions
        {
            WriteIndented = true
        });
    }
}
\end{lstlisting}

\section{Тестирование проекта}

Для проекта создан комплексный набор тестов, охватывающих различные аспекты системы:

\begin{itemize}
    \item \textbf{Модульные тесты} - проверяют отдельные компоненты системы
    \item \textbf{Интеграционные тесты} - проверяют взаимодействие нескольких компонентов
    \item \textbf{Тесты производительности} - оценивают эффективность работы кода
    \item \textbf{Краевые случаи} - проверяют работу системы в экстремальных ситуациях
\end{itemize}

Общее покрытие кода тестами составляет более 65\%, что соответствует поставленным требованиям.

\section{Выводы}

В рамках проекта "ВШЭ-Банк" была успешно разработана система учета личных финансов с применением современных принципов и паттернов проектирования. Проект демонстрирует:

\begin{itemize}
    \item Четкую структуру и разделение ответственности между компонентами
    \item Применение принципов SOLID и GRASP на практике
    \item Использование различных паттернов GoF для решения конкретных задач
    \item Высокую тестируемость и надежность кода
    \item Расширяемую архитектуру, готовую к добавлению новых функций
\end{itemize}

Разработанное приложение может служить хорошей основой для дальнейшего развития и расширения функциональности системы финансового учета.

\end{document}